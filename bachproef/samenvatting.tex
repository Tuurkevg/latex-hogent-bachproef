%%=============================================================================
%% Samenvatting
%%=============================================================================

% TODO: De "abstract" of samenvatting is een kernachtige (~ 1 blz. voor een
% thesis) synthese van het document.
%
% Een goede abstract biedt een kernachtig antwoord op volgende vragen:
%
% 1. Waarover gaat de bachelorproef?
% 2. Waarom heb je er over geschreven?
% 3. Hoe heb je het onderzoek uitgevoerd?
% 4. Wat waren de resultaten? Wat blijkt uit je onderzoek?
% 5. Wat betekenen je resultaten? Wat is de relevantie voor het werkveld?
%
% Daarom bestaat een abstract uit volgende componenten:
%
% - inleiding + kaderen thema
% - probleemstelling
% - (centrale) onderzoeksvraag
% - onderzoeksdoelstelling
% - methodologie
% - resultaten (beperk tot de belangrijkste, relevant voor de onderzoeksvraag)
% - conclusies, aanbevelingen, beperkingen
%
% LET OP! Een samenvatting is GEEN voorwoord!

%%---------- Nederlandse samenvatting -----------------------------------------
%
% TODO: Als je je bachelorproef in het Engels schrijft, moet je eerst een
% Nederlandse samenvatting invoegen. Haal daarvoor onderstaande code uit
% commentaar.
% Wie zijn bachelorproef in het Nederlands schrijft, kan dit negeren, de inhoud
% wordt niet in het document ingevoegd.

\IfLanguageName{english}{%
\selectlanguage{dutch}
\chapter*{Samenvatting}

\selectlanguage{english}
}{}

%%---------- Samenvatting -----------------------------------------------------
% De samenvatting in de hoofdtaal van het document

\chapter*{\IfLanguageName{dutch}{Samenvatting}{Abstract}}

Dit onderzoek richt zich op het evalueren van alternatieven voor VMware vCenter als managementplatformsysteem. Het onderzoek wordt uitgevoerd naar aanleiding van de beslissing van VMware (het bedrijf achter VMware vCenter) om de licentiekosten te verhogen. Als gevolg hiervan is Excentis op zoek naar een mogelijk alternatief voor VMware vCenter, aangezien de prijzen te hoog zijn geworden om nog rendabel te zijn voor Excentis. Door deze actie van VMware is niet alleen Excentis op zoek naar een alternatief, maar ook andere bedrijven.
De mogelijke alternatieven worden onderverdeeld in twee groepen: de open-source alternatieven en de betalende alternatieven (closed-source). Wat bieden de verschillende alternatieven aan en wat zijn de voor- en nadelen van deze alternatieven? De alternatieven worden beoordeeld op het gebied van high availability, de integratie met verschillende storage systemen, de ondersteunde hypervisors en de verschillen daartussen. Ook wordt er gekeken naar de integratie met andere systemen zoals Ansible, Nakivo en Foreman.
Alle mogelijke managementplatformen die in dit onderzoek worden bekeken, zullen worden geïnstalleerd en getest in de testomgeving van Excentis in de serverruimte. Hierbij zullen verschillende aspecten worden onderzocht,de bestaande hardware en ondersteuning, performance en stabiliteit, integratie met de bestaande omgeving en werking met verschillende storage systemen. Het onderzoek zal een volwaardig alternatief bieden voor het huidige systeem dat Excentis op dit moment gebruikt met VMware vCenter.