%%=============================================================================
%% Methodologie
%%=============================================================================

\chapter{\IfLanguageName{dutch}{Methodologie}{Methodology}}%
\label{ch:methodologie}

%% TODO: In dit hoofstuk geef je een korte toelichting over hoe je te werk bent
%% gegaan. Verdeel je onderzoek in grote fasen, en licht in elke fase toe wat
%% de doelstelling was, welke deliverables daar uit gekomen zijn, en welke
%% onderzoeksmethoden je daarbij toegepast hebt. Verantwoord waarom je
%% op deze manier te werk gegaan bent.
%% 
%% Voorbeelden van zulke fasen zijn: literatuurstudie, opstellen van een
%% requirements-analyse, opstellen long-list (bij vergelijkende studie),
%% selectie van geschikte tools (bij vergelijkende studie, "short-list"),
%% opzetten testopstelling/PoC, uitvoeren testen en verzamelen
%% van resultaten, analyse van resultaten, ...
%%
%% !!!!! LET OP !!!!!
%%
%% Het is uitdrukkelijk NIET de bedoeling dat je het grootste deel van de corpus
%% van je bachelorproef in dit hoofstuk verwerkt! Dit hoofdstuk is eerder een
%% kort overzicht van je plan van aanpak.
%%
%% Maak voor elke fase (behalve het literatuuronderzoek) een NIEUW HOOFDSTUK aan
%% en geef het een gepaste titel.

\subsection{Literatuurstudie en voorbereidend onderzoek}
Een objectief resultaat van alle managementplatformen is gewenst om deze vervolgens naast elkaar te kunnen vergelijken. Om dit te bereiken, wordt een vergelijkende studie uitgevoerd.De verschillende managementplatformen worden met elkaar vergeleken om te bepalen welke het beste aansluit bij de noden van Excentis.
Aan de hand van literatuurstudie en onderzoeken worden de managementplatformen geselecteerd die met elkaar worden vergeleken.
Er zal samen met Excentis worden overlopen welke eisen zeker voldaan moeten worden en wat het bedrijf echt nodig heeft in hun managementplatform, gecombineerd met hun bestaande infrastructuur. Er zal ook gekeken worden naar wat er momenteel allemaal wordt gebruikt binnen hun omgeving. Deze zullen dan worden meegenomen als basisvereisten voor de alternatieve managementplatformen.
Deze fase zal ook gebruikt worden om de testomgeving op te zetten en de hardware te voorzien die nodig is voor de testen.

\subsection{selecteren van managementplatformen met hun bijhorden hypervisor}
Er zijn verschillende soorten managementplatformen op de markt die kunnen worden getest en vergeleken. Hieruit moet er een bepaalde selectie komen waar er verder mee vergeleken kan worden voor in de proof of concept.
Aangezien de tijd en middelen beperkt zijn, is het belangrijk om hierop een goede minimumcriteria te hebben waarop de managementplatformen worden geselecteerd.

\begin{enumerate}
\item Minimaal 1 open-source managementplatform die geen betalende licentie heeft voor de software (exclusief support).
\item Er is een volwaardige supportaanbieding voor commercieel gebruik. Hierbij kan de klant mits een betalende licentie beroep doen op support bij problemen.
\item Het kan draaien op de bestaande DELL-servers die tot de beschikking zijn in de proof of concept.
\item Het managementplatform moet verschillende nodes kunnen aansturen en beheren.
\item Betalende managementplatformen die een licentie nodig hebben, bieden een gratis proefversie aan om deze studie te kunnen uitvoeren in de proof of concept-omgeving.
\end{enumerate}
Als deze criteria zijn voldaan voor een specifiek managementplatform, wordt deze geselecteerd voor te draaien in de proof of concept.

Microsoft SCVMM(Microsoft Hyper-v) ~\autocite{Eaton2019} is deprecated en wordt op dit moment niet meer actief ondersteund door Microsoft.
Proxmox VE~\autocite{Proxmox} is een open-source managementplatform dat geen betalende licentie heeft voor de software. Het biedt wel een betalende support aan voor commercieel gebruik. Dit platform kan draaien op de bestaande DELL-servers die tot de beschikking zijn in de proof of concept.
OpenStack ~\autocite{openstack2024} is ook een open-source managementplatform dat geen betalende licentie heeft voor de software. Het biedt ook een betalende support aan voor commercieel gebruik bij verschillende toepassingen. Dit platform kan ook draaien op de bestaande DELL-servers die tot de beschikking zijn in de proof of concept.
Xen Orchestra ~\autocite{el2021server} is ook een opensource virtual managementplatform dat geen betalende licentie heeft voor de software. Het biedt ook een betalende support aan voor commercieel gebruik bij verschillende toepassingen. Dit platform kan ook draaien op de bestaande DELL-servers die tot de beschikking zijn in de proof of concept.
Xen Center~\autocite{xencenter2024}  is closed-source maar bied wel een gratis proefversie aan. Ze bieden ook een enterprise licentie aan voor commercieel gebruik. Dit platform kan ook draaien op de bestaande DELL-servers die tot de beschikking zijn in de proof of concept.

\subsection{Geselecteerde managementplatformen (short list)} 
\begin{table}[h!]
    \centering
    \begin{tabular}{|l|l|l|l|}
    \hline
    \textbf{Platform} & \textbf{Type} & \textbf{Hypervisor} \\ \hline
    \hline
    Proxmox VE        & Open-source   & KVM,LXC             \\ \hline
    OpenStack         & Open-source   & KVM, Xen            \\ \hline
    Xen Orchestra  & Open-source & XCP-ng,Citrix Hypervisor(XenServer) \\ \hline
    Xen Center  & Closed-source & Citrix Hypervisor(XenServer) \\ \hline
    \hline
    \end{tabular}
    \caption{Vergelijking van virtualisatieplatforms}
\end{table}
\newpage
\subsection{Opzetten testomgeving/proof of concept}
De hardware en omgeving waarin alles wordt getest en opgebouwd, worden opgesteld in een testomgeving bij Excentis. Dit maakt het mogelijk om de managementplatformen te testen in een realistische omgeving, wat essentieel is om een objectief resultaat te bekomen.
Deze omgeving zal bestaan uit 2 tot 3 verouderde DELL servers die in de serverruimte van Excentis staan. Hierbij zal alles op kleine schaal kunnen worden uitgevoerd.
Op deze servers worden de geselecteerde managementplatformen geïnstalleerd.
Voor bepaalde situaties zal er een extra server gebruikt worden als er beperkingen zijn in bepaalde functionaliteiten van de managementplatformen. Dit kan bijvoorbeeld zijn dat er een minimum aantal nodes moet zijn om bepaalde functionaliteiten rond High Availability te kunnen testen.
\subsection{Uitvoeren onderzoek op testomgeving}
Voor de vergelijking van de managementplatformen worden de volgende acties uitgevoerd op de testomgeving van Excentis:
\begin{itemize}
\item werking met de bestaande infrastructuur van Excentis, vergeleken op prestatie en stabiliteit. (Hoe snel start een nieuwe virtuele machine op, wat zijn de minimum eisen voor de hardware, ...).
\item Testen en meten van de performance en stabiliteit van de bestaande direct attached storage,distributed storage en storage area netwerk systemen die Excentis gebruikt aan de hand van testdata op de nieuwe managementplatformen.
\item Individueele vergelijking van een distributed storage systeem in een virtual management cluster (1<nodes)
\item Prestatie en stabiliteit van de managementplatformen bij piekmomenten en failovers.
\item hot swapping van hard disks. Blijf alles correct werken? 
\item Neemt de ene node de virtuele machines over van de andere node bij een failover?
\end{itemize}

\subsection{Evalueren van de resultaten}
Deze acties worden stap voor stap uitgevoerd bij elk managementplatform, dit om een zo objectief mogelijk resultaat per managementplatform te verkrijgen.
Als alle acties zijn uitgevoerd, worden de resultaten met elkaar vergeleken en wordt er een conclusie getrokken over welk managementplatform het beste aansluit bij de noden van Excentis.