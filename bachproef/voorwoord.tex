%%=============================================================================
%% Voorwoord
%%=============================================================================

\chapter*{\IfLanguageName{dutch}{Woord vooraf}{Preface}}%
\label{ch:voorwoord}

%% TODO:
%% Het voorwoord is het enige deel van de bachelorproef waar je vanuit je
%% eigen standpunt (``ik-vorm'') mag schrijven. Je kan hier bv. motiveren
%% waarom jij het onderwerp wil bespreken.
%% Vergeet ook niet te bedanken wie je geholpen/gesteund/... heeft

Ik heb dit onderzoek uitgevoerd naar de interesse in hoe virtuele managementplatformen werken en hoe deze in productie bij bedrijven worden gebruikt.
Naar aanleiding van de veranderingen die VMWare heeft doorgevoerd, was dit het ideale moment om dit onderzoek te voeren en meer te weten te komen over deze systemen.
Ik heb heel veel bijgeleerd op het vlak van samenwerken met externe mensen, zoals mijn co-promotor en promotor.
Ik wil vooral de mensen bedanken die nauw betrokken waren bij dit onderzoek, namelijk doctor en lector Pieter-Jan Maenhaut als promotor met de nodige vakkennis. U was altijd direct bereikbaar en schoot meteen in actie wanneer ik vragen of feedback nodig had.
Verder wil ik ook mijn co-promotor Moreno Robyn bedanken, die me hielp met het faciliteren van de infrastructuur en de begeleiding doorheen het onderzoek.
Ik wil ook mijn vriendin Pauline De Vreese bedanken om mij te blijven motiveren om door te zetten, ondanks soms de moeilijke momenten die ik tegenkwam tijdens dit onderzoek.
Zeker toen ik vertraging opliep doordat er niet direct fysieke servers beschikbaar waren waarop ik mijn proof of concept kon draaien.
Dankzij deze ervaring heb ik ook leren onderzoeken, kritisch leren nadenken over systemen, leren refereren en wetenschappelijke artikelen leren lezen.
Ik hoop deze ervaring later in het werkveld zeker terug te kunnen gebruiken.