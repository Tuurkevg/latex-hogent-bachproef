%%=============================================================================
%% Voorwoord
%%=============================================================================

\chapter*{\IfLanguageName{dutch}{Woord vooraf}{Preface}}%
\label{ch:voorwoord}

%% TODO:
%% Het voorwoord is het enige deel van de bachelorproef waar je vanuit je
%% eigen standpunt (``ik-vorm'') mag schrijven. Je kan hier bv. motiveren
%% waarom jij het onderwerp wil bespreken.
%% Vergeet ook niet te bedanken wie je geholpen/gesteund/... heeft



Ik heb dit onderzoek uitgevoerd om meer te begrijpen over de werking van virtuele managementplatformen en hoe deze in productie worden gebruikt bij bedrijven.
Naar aanleiding van de veranderingen die VMware heeft doorgevoerd, was dit het ideale moment om dit onderzoek te verrichten en meer te leren over deze systemen.
Ik heb veel bijgeleerd, vooral op het gebied van samenwerken met externe mensen, zoals mijn co-promotor en promotor.

Ik wil in het bijzonder de mensen bedanken die nauw betrokken waren bij dit onderzoek, namelijk doctor en lector Pieter-Jan Maenhaut, mijn promotor, voor zijn vakkennis en begeleiding. U was altijd direct bereikbaar en schoot meteen in actie wanneer ik vragen had of feedback nodig had.
Verder wil ik ook mijn co-promotor, Moreno Robyn, bedanken voor zijn hulp bij het faciliteren van de infrastructuur en de begeleiding gedurende het onderzoek.
Ook wil ik mijn vriendin, Pauline De Vreese, bedanken voor haar voortdurende motivatie om door te zetten, zelfs tijdens de moeilijke momenten die ik tegenkwam tijdens dit onderzoek.

Mijn ouders, Serge Van Ginderachter en Sophie Gheysen, hebben mijn finale versie meermaals overlezen op zoek naar fouten of inhoudelijke incorrectheden. Hen wil ik hiervoor bedanken.
Zeker toen ik vertraging opliep doordat er niet direct fysieke servers beschikbaar waren waarop ik mijn proof of concept kon draaien.

Ik wil iedereen bedanken die mij ooit geholpen heeft om in te zien wat ik wilde bereiken of die heeft bijgedragen aan het starten van mijn studie Toegepaste Informatica aan HOGENT. Zonder hen zou ik waarschijnlijk hier niet zijn om deze bachelorproef te schrijven.

Dankzij deze ervaring heb ik ook leren onderzoeken, kritisch nadenken over systemen, leren refereren en wetenschappelijke artikelen leren lezen.

Het was een oprechte eer om deze bachelorproef te maken aan HOGENT. Ik zal voor eeuwig een HOGENTenaar blijven, in hart en ziel.
Ik hoop deze ervaring later in het werkveld zeker te kunnen gebruiken.