%%=============================================================================
%% Conclusie
%%=============================================================================

\chapter{Conclusie}%
\label{ch:conclusie}

% TODO: Trek een duidelijke conclusie, in de vorm van een antwoord op de
% onderzoeksvra(a)g(en). Wat was jouw bijdrage aan het onderzoeksdomein en
% hoe biedt dit meerwaarde aan het vakgebied/doelgroep? 
% Reflecteer kritisch over het resultaat. In Engelse teksten wordt deze sectie
% ``Discussion'' genoemd. Had je deze uitkomst verwacht? Zijn er zaken die nog
% niet duidelijk zijn?
% Heeft het onderzoek geleid tot nieuwe vragen die uitnodigen tot verder 
%onderzoek?

Het verwachte resultaat is een oplossing in de richting van de open-source wereld. Doordat VMware de prijzen significant verhoogt, is er een beweging ontstaan richting open-source oplossingen voor hypervisors en managementplatformen.
Hierbij ligt de focus specifiek op Proxmox VE als managementplatform met KVM als onderliggende hypervisor. Dit managementplatform is algemeen bekend en beschikt over een community met uitgebreide documentatie en support.
Het verwachte resultaat zal daarmee ook richting Proxmox VE gaan, aangezien het goedkoop in gebruik is en alvast goede ondersteuning biedt voor verschillende Linux-distributies.
De Windows-hypervisor Hyper-V biedt echter alleen ondersteuning op Windows, wat als een groot minpunt wordt beschouwd in een operationele omgeving bij Excentis.

Als dit onderzoek een objectief en goed resultaat oplevert, heeft Excentis een geschikte opvolging voor VMware, waarbij de kosten en nadelen in operationele werking zo minimaal mogelijk worden gehouden.
Hierbij wordt ook rekening gehouden met de tijd die het bedrijf kan besparen door een goede overgang naar een nieuwe managementplatformen.