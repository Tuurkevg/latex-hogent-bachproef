%%=============================================================================
%% Conclusie
%%=============================================================================

\chapter{Conclusie}%
\label{ch:conclusie}

% TODO: Trek een duidelijke conclusie, in de vorm van een antwoord op de
% onderzoeksvra(a)g(en). Wat was jouw bijdrage aan het onderzoeksdomein en
% hoe biedt dit meerwaarde aan het vakgebied/doelgroep? 
% Reflecteer kritisch over het resultaat. In Engelse teksten wordt deze sectie
% ``Discussion'' genoemd. Had je deze uitkomst verwacht? Zijn er zaken die nog
% niet duidelijk zijn?
% Heeft het onderzoek geleid tot nieuwe vragen die uitnodigen tot verder 
%onderzoek?

Het verwachte resultaat is een oplossing in de richting van de open-source wereld. Doordat VMware de prijzen significant verhoogt, is er een beweging ontstaan richting open-source oplossingen voor hypervisors en managementplatformen.
Hierbij ligt de focus specifiek op Proxmox VE als managementplatform met KVM als onderliggende hypervisor. Dit managementplatform is algemeen bekend en beschikt over een community met uitgebreide documentatie en support.
Het verwachte resultaat zal daarmee ook richting Proxmox VE gaan, aangezien het goedkoop in gebruik is en alvast goede ondersteuning biedt voor verschillende Linux-distributies.
De Windows-hypervisor Hyper-V biedt echter alleen ondersteuning op Windows, wat als een groot minpunt wordt beschouwd in een operationele omgeving bij Excentis.

Als dit onderzoek een objectief en goed resultaat oplevert, heeft Excentis een geschikte opvolging voor VMware, waarbij de kosten en nadelen in operationele werking zo minimaal mogelijk worden gehouden.
Hierbij wordt ook rekening gehouden met de tijd die het bedrijf kan besparen door een goede overgang naar een nieuwe managementplatformen.

Uit de proof of concept kan het volgende worden geconcludeerd.
\section{Storage integratie}
\begin{itemize}
\item Proxmox VE biedt ondersteuning voor zowel distributed storage met CEPH als SAN met iSCSI. Beide systemen zijn eenvoudig configureerbaar en dienen zo goed als niet buiten de webintegratie te gaan.
\item Xen Orchestra biedt ook voor beide systemen een integratie. Voor distributed storage met XOSTOR is er iets meer configureerwerk dan het systeem dat Proxmox biedt. Verder werkt het systeem zeer correct zoals er verwacht kan worden van distributed storage.
\item XenCenter heeft alleen de mogelijkheid om te functioneren met SAN iSCSI. Het biedt geen ingebouwd systeem voor distributed storage dat officieel ondersteund wordt.
\end{itemize}

\section{High Availability integratie}
Alle 3 de virtuele managementplatformen bieden High Availability integratie. Die werkt zoals bij VMWare. Ze bieden alle 3 een eenvoudige configuratie aan via de ingebouwde omgeving.
Proxmox VE biedt zelf een lichtere verbetering aan op vlak van snelheid bij het herkennen van een node die een failure heeft. Hierdoor is het eindresultaat dat de virtuele machine bij Proxmox VE het snelst terug online zal zijn.
Verder kan ook worden opgemerkt dat de virtuele machines met als storage systeem distributed storage het snelst recupereren na een failover, voor zowel Proxmox VE als Xen Orchestra.

\section{Migratie virtuele machines tussen fysieke nodes}
Hier kan ook geconcludeerd worden dat alle 3 onderzochte systemen een eenvoudige, goed werkende integratie hebben voor het migreren van een virtuele machine tussen 2 verschillende fysieke nodes.
Dit kan handig zijn bij een gepland onderhoud of bij het stopzetten van een verouderde server die vervangen gaat worden. Hierbij kan er bij alle 3 de systemen een oplossing aangeboden worden zodat de downtime van kritieke virtuele machines zo weinig mogelijk is.
Hier valt niet direct een betere prestatie tussen de 3 systemen op. Wel dat distributed storage veel sneller is dan de externe SAN-oplossing met iSCSI.

\section{Hot swapping van schijven}
Het kan soms nodig zijn om fysieke schijven op de node te vervangen waar er bijvoorbeeld een distributed storage oplossing op draait. Om geen data te verliezen of geen downtime te hebben is het cruciaal om in noodgevallen eenvoudig een schijf te kunnen hot swappen.
Hierbij is het belangrijk dat de gebruiker zo eenvoudig mogelijk dit kan doen, waarbij liefst niet uit de GUI/applicatie moet worden gegaan.
Proxmox VE biedt hiervoor de beste oplossing, waarbij de lijst van schijven te allen tijde direct wordt geüpdatet bij het verwisselen van de schijf en de mogelijkheid biedt om direct de schijf in te stellen met bijvoorbeeld CEPH voor distributed storage zonder een restart of uitschakelen van een virtuele machine of server.
\newpage
\section{Eindconclusie}
Als eindconclusie van heel dit onderzoek kan Proxmox VE als beste uit de bus worden gehaald. Proxmox heeft bij de specifieke systemen die getest zijn geen gebreken getoond. Het biedt alles dat Excentis specifiek nodig heeft.
Verder heeft Proxmox VE ook enterprise-oplossingen waar nieuwere en verbeterde technologieën inzitten en een 24/7 support inbegrepen is mocht dat nodig zijn. Met al deze factoren in overweging te nemen kan deze conclusie duidelijk worden gemaakt.
Xen Orchestra biedt ook zo goed als alles aan dat nodig is voor Excentis. Het grootste verschil is wel dat Proxmox VE al onze geteste systemen aanbiedt in de gratis communityversie en Xen Orchestra biedt bijvoorbeeld alleen distributed storage aan bij de enterprise-versie.
XenCenter geeft meer de blik dat het verouderd is en vaak het gevoel heeft om de GUI-applicatie te verlaten en toch in de CLI fysiek op de server te gaan om wijzigingen door te voeren. Het biedt ook geen gratis versie aan waardoor het verplicht is om direct een enterprise-licentie te nemen als er gekozen wordt om met dit systeem te werken.