%==============================================================================
% Sjabloon onderzoeksvoorstel bachproef
%==============================================================================
% Gebaseerd op document class `hogent-article'
% zie <https://github.com/HoGentTIN/latex-hogent-article>

% Voor een voorstel in het Engels: voeg de documentclass-optie [english] toe.
% Let op: kan enkel na toestemming van de bachelorproefcoördinator!
\documentclass{hogent-article}

% Invoegen bibliografiebestand
\addbibresource{voorstel.bib}

% Informatie over de opleiding, het vak en soort opdracht
\studyprogramme{Professionele bachelor toegepaste informatica}
\course{Bachelorproef}
\assignmenttype{Onderzoeksvoorstel}
% Voor een voorstel in het Engels, haal de volgende 3 regels uit commentaar
% \studyprogramme{Bachelor of applied information technology}
% \course{Bachelor thesis}
% \assignmenttype{Research proposal}

\academicyear{2024-2025} % TODO: pas het academiejaar aan

% TODO: Werktitel
\title{Evaluatie van Alternatieven voor VMware: Een Onderzoek naar Hypervisors met Ondersteuning voor Nakivo, Ansible, en Foreman binnen Excentis}

% TODO: Studentnaam en emailadres invullen
\author{Arthur Van Ginderachter}
\email{arthur.vanginderachter@student.hogent.be}

% TODO: Medestudent
% Gaat het om een bachelorproef in samenwerking met een student in een andere
% opleiding? Geef dan de naam en emailadres hier
% \author{Yasmine Alaoui (naam opleiding)}
% \email{yasmine.alaoui@student.hogent.be}

% TODO: Geef de co-promotor op 
\supervisor[Co-promotor]{M. Robyn (Excentis, \href{mailto:moreno.robyn@excentis.com}{moreno.robyn@excentis.com})}

% Binnen welke specialisatierichting uit 3TI situeert dit onderzoek zich?
% Kies uit deze lijst:
%
% - Mobile \& Enterprise development
% - AI \& Data Engineering
% - Functional \& Business Analysis
% - System \& Network Administrator
% - Mainframe Expert
% - Als het onderzoek niet past binnen een van deze domeinen specifieer je deze
%   zelf
%
\specialisation{Systeem \& Netwerken}
\keywords{VMware, Hypervisors}

\begin{document}

\begin{abstract}
  Hier schrijf je de samenvatting van je voorstel, als een doorlopende tekst van één paragraaf. Let op: dit is geen inleiding, maar een samenvattende tekst van heel je voorstel met inleiding (voorstelling, kaderen thema), probleemstelling en centrale onderzoeksvraag, onderzoeksdoelstelling (wat zie je als het concrete resultaat van je bachelorproef?), voorgestelde methodologie, verwachte resultaten en meerwaarde van dit onderzoek (wat heeft de doelgroep aan het resultaat?).
\end{abstract}

\tableofcontents
% De hoofdtekst van het voorstel zit in een apart bestand, zodat het makkelijk
% kan opgenomen worden in de bijlagen van de bachelorproef zelf.
%---------- Inleiding ---------------------------------------------------------

% TODO: Is dit voorstel gebaseerd op een paper van Research Methods die je
% vorig jaar hebt ingediend? Heb je daarbij eventueel samengewerkt met een
% andere student?
% Zo ja, haal dan de tekst hieronder uit commentaar en pas aan.

%\paragraph{Opmerking}

% Dit voorstel is gebaseerd op het onderzoeksvoorstel dat werd geschreven in het
% kader van het vak Research Methods dat ik (vorig/dit) academiejaar heb
% uitgewerkt (met medesturent VOORNAAM NAAM als mede-auteur).
% 

\section{Inleiding}%
\label{sec:inleiding}

Door de aankondiging van VMware om de licentiekosten te verhogen~\autocite{device42_2024}, zijn er bedrijven die beginnen te denken aan een eventueel alternatief voor VMware ESXi/VMWare vCenter.
De prijzen voor deze software zijn zo hoog~\autocite{Hale2024} dat bedrijven zoeken naar alternatieven.
Vaak gaat het om bedrijven die virtualisatie hebben als een service, waarbij dit niet de allerbelangrijkste zorg is binnen het bedrijf.
VMware-producten zit hard vastgeworteld in bedrijven doordat ze werken met verschillende automatisatietools en scripts. Zo'n overgang naar een nieuwe managementplatform zal een impact hebben op elke IT-infrastructuur.
Om deze overgang ook vloeiend te kunnen laten lopen, moet er in de managementplatform ook ondersteuning zijn voor deze verschillende tools.
De mensen binnen IT moeten hierbij ook een omscholing krijgen of zich gaan verdiepen in de managementplatformen  voor virtualisatie. De documentatie en support/community achter de managementplatformen is zeker een aspect dat in rekening moet worden genomen.
Deze bachelorproef richt zich op het bedrijf Excentis: Welke alternatieven zijn er om het huidige VMware vCenter te vervangen en hoe kan die passen binnen de noden van omgeving van Excentis?
De volgende vragen moeten wij ons stellen bij het zoeken naar alternatieven op de markt.
\begin{enumerate}
\item Welke functionele en prestatieverschillen zijn er tussen open-source en closed-source managementplatformen, en welke voor- en nadelen hebben deze verschillen voor het bedrijf Excentis?
\item In welke mate integreren managementplatformen met tools zoals Nakivo, Ansible en Foreman, en hoe kunnen deze toegepast worden binnen Excentis?
\item Hoe presteren managementplatformen op het gebied van High Availability(failover,\newline schaalbaarheid,backups,...)?
\item Hoe kan de bestaande ondersteuning van Direct Attached Storage en Serial Attached SCSI in VMWare vCenter worden overgezet naar een alternatief managementplatformen binnen de infrastructuur van Excentis?
\item Op welke manier is er een mogelijkheid om de bestaande managementplatformen omgeving van Excentis over te zetten naar het nieuwe alternatief?
\end{enumerate}



% Waarover zal je bachelorproef gaan? Introduceer het thema en zorg dat volgende zaken zeker duidelijk aanwezig zijn:

% \begin{itemize}
%   \item kaderen thema
%   \item de doelgroep
%   \item de probleemstelling en (centrale) onderzoeksvraag
%   \item de onderzoeksdoelstelling
% \end{itemize}

% Denk er aan: een typische bachelorproef is \textit{toegepast onderzoek}, wat betekent dat je start vanuit een concrete probleemsituatie in bedrijfscontext, een \textbf{casus}. Het is belangrijk om je onderwerp goed af te bakenen: je gaat voor die \textit{ene specifieke probleemsituatie} op zoek naar een goede oplossing, op basis van de huidige kennis in het vakgebied.

% De doelgroep moet ook concreet en duidelijk zijn, dus geen algemene of vaag gedefinieerde groepen zoals \emph{bedrijven}, \emph{developers}, \emph{Vlamingen}, enz. Je richt je in elk geval op it-professionals, een bachelorproef is geen populariserende tekst. Eén specifiek bedrijf (die te maken hebben met een concrete probleemsituatie) is dus beter dan \emph{bedrijven} in het algemeen.

% Formuleer duidelijk de onderzoeksvraag! De begeleiders lezen nog steeds te veel voorstellen waarin we geen onderzoeksvraag terugvinden.

% Schrijf ook iets over de doelstelling. Wat zie je als het concrete eindresultaat van je onderzoek, naast de uitgeschreven scriptie? Is het een proof-of-concept, een rapport met aanbevelingen, \ldots Met welk eindresultaat kan je je bachelorproef als een succes beschouwen?

%---------- Stand van zaken ---------------------------------------------------

\section{Literatuurstudie}
\label{sec:literatuurstudie}
VMware~\autocite{vmware} is een bedrijf die zich bezighoud met alles rond virtualisatie. De sterke prijsstijgingen van VMware zorgen ervoor dat veel bedrijven afhaken en op zoek gaan naar alternatieven~\autocite{Hale2024}. Er zijn verschillende alternatieve managementplatformen met bijhorende hypervisors.

\subsection{Hypervisors}
In het huidige systeem dat Excentis gebruikt als onderliggen hypervisor is VMWARE ESXI~\autocite{vmware}. Deze is closed source en werkt binnen het VMware systeem.
Een voorbeeld hiervan is 'KVM' (Kernel-based Virtual Machine)\autocite{KVM}. KVM is open source en vrij te gebruiken voor iedereen\autocite{KVM}. Microsoft Hyper-V ~\autocite{Eaton2019} wordt ook genoemd als mogelijke vergelijking met VMware ESXI~\autocite{fayyad2013benchmarking}. Hierin wordt een vergelijking gemaakt tussen verschillende hypervisors die geselecteerd zijn.
Verder in de virtualisatie wereld hebben wij ook nog het Xen Project~\autocite{xenproject}. Dit is een open-source hypervisor die zich vooral richt op cloud computing en server virtualisatie ~\autocite{binu2011virtualization}.
XenServer~\autocite{xenserver} is ook een alternatieve software keuze voor VMware ESXI. XenServer is een commerciële hypervisor die zich richt op bedrijven en enterprise-ondersteuning biedt.

\subsection{managementplatformen}
Excentis gebruikt bovenliggend als managementplatformen VMWare vCenter~\autocite{vmware}. Deze zou dus moeten vervangen worden.
Proxmox VE~\autocite{Proxmox} is een open-source managementplatform dat werkt met KVM en enterprise-ondersteuning aanbiedt voor bedrijven. Uit dit onderzoek~\autocite{ally2018comparative} blijkt dat Proxmox, dat gebruik maakt van KVM zeker niet onderdoet tegenover andere closed source systemen.
OpenStack is een open-source cloud computing platform~\autocite{openstack2024}. Deze bied niet alleen ondersteuning voor KVM maar ook voor Xen~\autocite{oleksiuk2023comparative}.
Microsoft System Center Virtual Machine Manager (SCVMM)~\autocite{microsoftvmm2025} is van microsoft en bied een managementplatform systeem aan voor Hyper-V.
XenCenter~\autocite{xencenter2024} bied een managementplatform aan voor XenServer~\autocite{xenserver}.

\subsection{Open source systemen}
Er zijn vele soorten softwarepakketten op de markt die hypervisors en managementplatformen aanbieden. Hieruit kunnen wij ze onderverdelen in twee groepen: de closed-source en open-source hypervisors. Elke groep heeft zijn voor- en nadelen. Alles hangt steeds af van de noden en vereisten van het bedrijf of instelling.
Proxmox VE is open source~\autocite{Proxmox}. Dit wil zeggen dat de systemen kosteloos gebruikt kunnen worden. Support bij problemen valt hierbij wel af. Er is wel een mogelijkheid om bij hen een enterprise support services subscription te nemen. Hierbij kan er alsnog beroep worden gedaan bij eventuele problemen.
Open source software is een evidentie maar zorgt voor meer zelfstandig werk bij installatie en rust meer op de community achter de software.

\subsection{closed source systemen}
Andere alternatieven systemen zijn closed-source. Hierbij hebben we het onder andere over VMware ESXi/VMware vCenter~\autocite{vmware} en Microsoft Hyper-V~\autocite{Eaton2019}. Deze software systemen zijn niet gratis en vragen een licentie om te mogen gebruiken. Dit is een nadeel ten opzichte van open-source systemen.
Bij bedrijven waar stabiliteit en betrouwbaarheid hoog in het vaandel staan, zoals in een wetenschappelijke omgeving, kan het een goede keuze zijn om te kiezen voor closed-source hypervisors~\autocite{voras2012early}. Dit wil niet zeggen dat andere systemen uitgesloten zijn.
\subsection{Soorten storage systemen}
Wanneer we praten over storage in de virtualisatiewereld, dan hebben we het ook over 'Serial-Attached SCSI' (SAS) en 'Direct-Attached Storage' (DAS). Deze technologieën zorgen ervoor dat er een hoge beschikbaarheid is van de data en dat deze snel kan worden opgevraagd~\autocite{griswold2002storage}. Dit is een belangrijk aspect in de virtualisatiewereld en is een belangrijke eis voor Excentis.
SAS is een snelle, betrouwbare opslaginterface die servers en high-performance opslagapparaten verbindt~\autocite{aravindan2014performance}. Dit is een belangrijk om een consistente oplsag te garanderen. DAS is een opslagarchitectuur waarbij opslag direct fysiek is verbonden aan een enkele server, zonder tussenkomst van een netwerk. Deze technologie is goedkoper en eenvoudiger, maar garandeert niet alle voordelen die SAS wel kan garanderen~\autocite{griswold2002storage}.
DAS wordt vaak gebruikt in kleinere bedrijven waar de data niet zo belangrijk is en waar de data niet zo vaak wordt opgevraagd. SAS wordt vaak gebruikt in grotere bedrijven waar de data zeer belangrijk is en waar de data vaak wordt opgevraagd~\autocite{griswold2002storage}.
Zo goed als alle managementplatformen ondersteunen DAS en SAS in ons onderzoek. Excentis wilt weten hoe deze ondersteuning kan worden overgezet naar een nieuw managementplatform.

\subsection{High Availability-ondersteuning}
In dit onderzoek~\autocite{dudnik2017creating} wordt een grote vergelijking gedaan tussen hypervisors en hun managementplatformen. Hierin spit men toe op bepaalde aspecten binnen de term High Availability.
Er zijn verschillende aspecten die moeten worden bekeken en waaraan voldaan moet worden om aan een goede High Availability te voldoen. Een failover-systeem dat het ene systeem het andere systeem over laat nemen bij een bepaalde fout, is een van die aspecten in het onderzoek van~\autocite{dudnik2017creating}.
Op netwerkniveau moet er ook nagedacht worden over zowel schaalbaarheid bij piekmomenten als bij problemen die zich voordoen in het netwerk. Migratie tussen verschillende fysieke servers moet ook mogelijk zijn om een goede High Availability te garanderen~\autocite{dudnik2017creating}.
Workloadmanagers kunnen een goede oplossing zijn om overbelasting tegen te gaan op drukke piekmomenten. Back-ups van de data worden ook gezien als een belangrijk aspect voor High Availability.
% \subsection{KVM}
% \subsection{Microsoft Hyper-V}
% \subsection{Kernel-based Virtual Machine}
% \subsection{ProxMox VE}
% \subsection{xenproject}
% \subsection{High Availability-ondersteuning}
% \subsection{Direct Attached Storage en Serial Attached SCSI}

% Voor literatuurverwijzingen zijn er twee belangrijke commando's:
% \autocite{KEY} => (Auteur, jaartal) Gebruik dit als de naam van de auteur
%   geen onderdeel is van de zin.
% \textcite{KEY} => Auteur (jaartal)  Gebruik dit als de auteursnaam wel een
%   functie heeft in de zin (bv. ``Uit onderzoek door Doll & Hill (1954) bleek
%   ...'')


%---------- Methodologie ------------------------------------------------------
\section{Methodologie}%
\label{sec:methodologie}
Een objectief resultaat van alle managementplatformen is gewenst om deze vervolgens naast elkaar te kunnen vergelijken. Om dit te bereiken, wordt een vergelijkende studie uitgevoerd. De verschillende managementplatformen worden met elkaar vergeleken om te bepalen welke het beste aansluit bij de noden van Excentis.
Aan de hand van literatuurstudie en onderzoeken worden de managementplatformen geselecteerd die met elkaar worden vergeleken.
Er zal samen met Excentis worden overlopen welke eisen zeker voldaan moeten worden en wat het bedrijf echt nodig heeft in hun managementplatform, gecombineerd met hun bestaande infrastructuur. Er zal ook gekeken worden naar wat er momenteel allemaal wordt gebruikt binnen hun omgeving. Deze zullen dan worden meegenomen als basisvereisten voor de alternatieve managementplatformen.
De hardware en omgeving waarin alles wordt getest en opgebouwd, worden opgesteld in een testomgeving bij Excentis. Dit maakt het mogelijk om de managementplatformen te testen in een realistische omgeving, wat essentieel is om een objectief resultaat te bekomen.
Deze omgeving zal bestaan uit 2 verouderde DELL servers die in de serverruimte van Excentis staan. Hierbij zal alles op kleine schaal kunnen worden uitgevoerd.
Op deze servers worden de geselecteerde managementplatformen geïnstalleerd.
Voor de vergelijking van de managementplatformen worden de volgende acties uitgevoerd op de testomgeving van Excentis:
\begin{itemize}
\item werking met de bestaande hardware van Excentis, vergeleken op prestatie en stabiliteit. (Hoe snel start een nieuwe virtuele machine op, wat zijn de minimum eisen voor de hardware, ...).
\item Testen en meten van de performance en stabiliteit van de bestaande Direct Attached Storage en Serial Attached SCSI-systemen die Excentis gebruikt aan de hand van testdata op de nieuwe managementplatformen.
\item De integratie van bepalende systemen binnen Excentis testen, zoals CI/CD-tools, op de nieuwe managementplatformen.
\item Prestatie en stabiliteit van de managementplatformen bij piekmomenten en failovers.
\end{itemize}
Deze acties worden stap voor stap gedaan bij elke managementplatform. Deze worden gedaan om zo een zo objectief mogelijk resultaat te krijgen per managementplatform.
Als alle acties zijn uitgevoerd, worden de resultaten met elkaar vergeleken en wordt er een conclusie getrokken over welk managementplatform het beste aansluit bij de noden van Excentis.
% Hier beschrijf je hoe je van plan bent het onderzoek te voeren. Welke onderzoekstechniek ga je toepassen om elk van je onderzoeksvragen te beantwoorden? Gebruik je hiervoor literatuurstudie, interviews met belanghebbenden (bv.~voor requirements-analyse), experimenten, simulaties, vergelijkende studie, risico-analyse, PoC, \ldots?

% Valt je onderwerp onder één van de typische soorten bachelorproeven die besproken zijn in de lessen Research Methods (bv.\ vergelijkende studie of risico-analyse)? Zorg er dan ook voor dat we duidelijk de verschillende stappen terug vinden die we verwachten in dit soort onderzoek!

% Vermijd onderzoekstechnieken die geen objectieve, meetbare resultaten kunnen opleveren. Enquêtes, bijvoorbeeld, zijn voor een bachelorproef informatica meestal \textbf{niet geschikt}. De antwoorden zijn eerder meningen dan feiten en in de praktijk blijkt het ook bijzonder moeilijk om voldoende respondenten te vinden. Studenten die een enquête willen voeren, hebben meestal ook geen goede definitie van de populatie, waardoor ook niet kan aangetoond worden dat eventuele resultaten representatief zijn.

% Uit dit onderdeel moet duidelijk naar voor komen dat je bachelorproef ook technisch voldoen\-de diepgang zal bevatten. Het zou niet kloppen als een bachelorproef informatica ook door bv.\ een student marketing zou kunnen uitgevoerd worden.

% Je beschrijft ook al welke tools (hardware, software, diensten, \ldots) je denkt hiervoor te gebruiken of te ontwikkelen.

% Probeer ook een tijdschatting te maken. Hoe lang zal je met elke fase van je onderzoek bezig zijn en wat zijn de concrete \emph{deliverables} in elke fase?

%---------- Verwachte resultaten ----------------------------------------------
\section{Verwacht resultaat, conclusie}%
\label{sec:verwachte_resultaten}

Het verwachte resultaat is een oplossing in de richting van de open-source wereld. Doordat VMware de prijzen significant verhoogt, is er een beweging ontstaan richting open-source oplossingen voor hypervisors en managementplatformen.
Hierbij ligt de focus specifiek op Proxmox VE als managementplatform met KVM als onderliggende hypervisor. Deze managementplatform is algemeen bekend en beschikt over een community met uitgebreide documentatie en support.
Het verwachte resultaat zal daarmee ook richting Proxmox VE gaan, aangezien het goedkoop in gebruik is en alvast goede ondersteuning biedt voor verschillende Linux-distributies.
De Windows-hypervisor Hyper-V biedt echter alleen ondersteuning op Windows, wat als een groot minpunt wordt beschouwd in een operationele omgeving bij Excentis.

Als dit onderzoek een objectief en goed resultaat oplevert, heeft Excentis een geschikte opvolging voor VMware, waarbij de kosten en nadelen in operationele werking zo minimaal mogelijk worden gehouden.
Hierbij wordt ook rekening gehouden met de tijd die het bedrijf kan besparen door een goede overgang naar een nieuwe managementplatformen.
% Hier beschrijf je welke resultaten je verwacht. Als je metingen en simulaties uitvoert, kan je hier al mock-ups maken van de grafieken samen met de verwachte conclusies. Benoem zeker al je assen en de onderdelen van de grafiek die je gaat gebruiken. Dit zorgt ervoor dat je concreet weet welk soort data je moet verzamelen en hoe je die moet meten.

% Wat heeft de doelgroep van je onderzoek aan het resultaat? Op welke manier zorgt jouw bachelorproef voor een meerwaarde?

% Hier beschrijf je wat je verwacht uit je onderzoek, met de motivatie waarom. Het is \textbf{niet} erg indien uit je onderzoek andere resultaten en conclusies vloeien dan dat je hier beschrijft: het is dan juist interessant om te onderzoeken waarom jouw hypothesen niet overeenkomen met de resultaten.


\newpage  % Dwingt LaTeX om naar een nieuwe pagina te gaan
\printbibliography[heading=bibintoc]

\end{document}